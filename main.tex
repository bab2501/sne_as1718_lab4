%%%%%%%%%%%%%%%%%%%%%%%%%%%%%%%%%%%%%%%%%
% Beamer Presentation
% LaTeX Template
% Version 1.0 (10/11/12)
%
% This template has been downloaded from:
% http://www.LaTeXTemplates.com
%
% License:
% CC BY-NC-SA 3.0 (http://creativecommons.org/licenses/by-nc-sa/3.0/)
%
%%%%%%%%%%%%%%%%%%%%%%%%%%%%%%%%%%%%%%%%%

%----------------------------------------------------------------------------------------
%	PACKAGES AND THEMES
%----------------------------------------------------------------------------------------

\documentclass[10pt]{beamer}
\mode<presentation> {

% The Beamer class comes with a number of default slide themes
% which change the colors and layouts of slides. Below this is a list
% of all the themes, uncomment each in turn to see what they look like.

%\usetheme{default}
%\usetheme{AnnArbor}
%\usetheme{Antibes}
%\usetheme{Bergen}
%\usetheme{Berkeley}����JFIF����C
%\usetheme{Berlin}
%\usetheme{Boadilla}
%\usetheme{CambridgeUS}
%\usetheme{Copenhagen}
%\usetheme{Darmstadt}
%\usetheme{Dresden}
%\usetheme{Frankfurt}
%\usetheme{Goettingen}
%\usetheme{Hannover}
%\usetheme{Ilmenau}
%\usetheme{JuanLesPins}
%\usetheme{Luebeck}
%\usetheme{Madrid}
%\usetheme{Malmoe}
%\usetheme{Marburg}
%\usetheme{Montpellier}
\usetheme{PaloAlto}
%\usetheme{Pittsburgh}
%\usetheme{Rochester}
%\usetheme{Singapore}
%\usetheme{Szeged}
%\usetheme{Warsaw}

% As well as themes, the Beamer class has a number of color themes
% for any slide theme. Uncomment each of these in turn to see how it
% changes the colors of your current slide theme.

%\usecolortheme{albatross}
%\usecolortheme{beaver}
%\usecolortheme{beetle}
%\usecolortheme{crane}
%\usecolortheme{dolphin}
%\usecolortheme{dove}
%\usecolortheme{fly}
%\usecolortheme{lily}
%\usecolortheme{orchid}
%\usecolortheme{rose}
%\usecolortheme{seagull}
%\usecolortheme{seahorse}
\usecolortheme{whale}
%\usecolortheme{wolverine}

%\setbeamertemplate{footline} % To remove the footer line in all slides uncomment this line
\setbeamertemplate{footline}[page number] % To replace the footer line in all slides with a simple slide count uncomment this line

%\setbeamertemplate{navigation symbols}{} % To remove the navigation symbols from the bottom of all slides uncomment this line
}

\usepackage{graphicx} % Allows including images
\usepackage{booktabs} % Allows the use of \toprule, \midrule and \bottomrule in tables
\usepackage{amsmath}
\usepackage{subfig}
\usepackage{caption}
\usepackage{mathabx}
\usepackage{wasysym}
\usepackage{wrapfig}
\usepackage{tikz}
\usepackage{animate}
\usepackage{minted}
\usepackage{listings}
\usepackage{listings}
\usetikzlibrary{shapes.geometric, arrows}
\usepackage{minted}
\usepackage{color}

\usepackage{xcolor}
\usepackage{listings}
\lstset{basicstyle=\ttfamily,
  showstringspaces=false,
  commentstyle=\color{red},
  keywordstyle=\color{blue}
}

%\usepackage{subcaption}
%----------------------------------------------------------------------------------------
%	TITLE PAGE
%----------------------------------------------------------------------------------------
\title[]{Lab \#4: Wi-Fi deauthentication attack \\  } % The short title appears at the bottom of every slide, the full title is only on the title page

\author{Alexander Blaauwgeers} % Your name
\institute[University of Amsterdam] % Your institution as it will appear on the bottom of every slide, may be shorthand to save space
{
University of Amsterdam \\ % Your institution for the title page
\medskip
%\textit{john@smith.com} % Your email address
}
\date{Mei 16, 2018} % Date, can be changed to a custom date

\begin{document}

\begin{frame}
\titlepage % Print the title page as the first slide
\end{frame}

%----------------------------------------------------------------------------------
% \section{Introduction }
% \begin{frame}{Introduction}
% \begin{block}{Question}
% There are radio stations that send out RDS information so you know traffic jams are ahead or know the name of the
% song that is currently playing.
% \begin{itemize}
%     \item Is there a way to retrieve this information using SDR?
%     \item What kind of information can you get out of this?
%     \item How does the protocol work?
% \end{itemize}
% \end{block}
% \end{frame}

\begin{frame}{What is RDS?}
Radio Data System

\begin{itemize}
    \item Allows to send data over FM radio
    \item Multiple fields
    \begin{itemize}
        \item Most only flags or short identifiers
    \end{itemize}
    \item Radio Text field
    \begin{itemize}
        \item Allows for arbitrary 64-character strings
    \end{itemize}
    \item Traffic Message Channel
    \begin{itemize}
        \item Might be encoded
    \end{itemize}
\end{itemize}
\end{frame}

%----------------------------------------------------------------------------------
\section{Theory}
%----------------------------------------------------------------------------------
\begin{frame}{Theory}
\centering
\begin{itemize}
    \item Data is carried on a 57kHz subcarrier
\end{itemize}

\includegraphics[width=\textwidth]{fmspectrum.png}

\end{frame}

%----------------------------------------------------------------------------------
\section{Receiving RDS}
\begin{frame}[fragile]{Receiving RDS Signals}
\begin{itemize}
    \item GQRX allows to receive RDS channel information natively
    \begin{itemize}
        \item But only basic and interpreted information
    \end{itemize}
    \item \verb|redsea|
    \begin{itemize}
        \item Low level tool that can extract all raw RDS fields from \verb|rtl_fm| input
    \end{itemize}
\end{itemize}
\end{frame}

%----------------------------------------------------------------------------------
\section{Transmitting RDS}

\begin{frame}{Security}
    \pause \centering None
\end{frame}

\begin{frame}{Transmitting RDS Signals}

\begin{itemize}
    \item It is possible to send FM/RDS signals with your own transmitter
    \begin{itemize}
        \item Possible to make users switch channels with the appropriate flags
        \item Display arbitrary text on user radio displays
        \item Let navigation equipment take detours
        \pause \item But it is illegal (except below 50 nW) \footnote{\tiny{Subcategorie 8.C "Regeling gebruik van frequentieruimte zonder vergunning en zonder meldingsplicht 2015"}}
    \end{itemize}
\end{itemize}



\end{frame}

%----------------------------------------------------------------------------------
\section{PoC - 102.3 MHz}
\begin{frame}{Hardware}
    \includegraphics[width=9cm]{FMzend.png}
\end{frame}

\begin{frame}{Code}
    \includegraphics[width=9cm]{FMcode.png}
\end{frame}

\begin{frame}{102.3 MHz}
    \includegraphics[width=9cm]{photo_2018-04-20_10-12-16.jpg}
\end{frame}

%----------------------------------------------------------------------------------

%----------------------------------------------------------------------------------

\begin{frame}{Questions?}

Questions?

%\def\newblock{}
%\bibliographystyle{unsrt}
%\bibliography{mybib}
\end{frame}

\end{document} 
%---------------------------------------------------------------------------
